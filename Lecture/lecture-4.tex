\section{Lecture 4 - 09/14/2022}

\subsection{Holomorphic Implies Infinitely Differentiable}

\begin{example}
Let $\gamma$ be a path from $i$ to $2$, then
\[\int_\gamma z^5 dz = \frac{1}{6} z^6 |_i^2\]
\end{example}

\begin{definition}
We say $f$ has anti-derivative (otherwise known as \textbf{primitive}) if $F'(z) = f(z)$.
\end{definition}

\begin{proposition}
    If $f$ is primitive, then
    \[\int_\gamma f(z) dz = F(z_1) - F(z_0)\]
    , where $z_0$ is the start of $\gamma$ and $z_1$ is the end of $\gamma$. In particular if $z_0 = z_1$,
    \[\int_\gamma f(z) dz = 0\]
\end{proposition}

\begin{proof}
Chain-Rule and Fundamental Theorem of Calculus
\end{proof}

Recall Cauchy's Integral Formula:
\[f(z) = \frac{1}{2\pi i} \oint_{\partial G} \frac{f(\xi)}{\xi - z} d\xi\]

We claim that, taking the derivative of $f(z)$ gives
\[f'(z) = \frac{1}{2\pi i}  \oint_{\partial G} \frac{1}{(\xi - z)^2} f(\xi) d\xi\]

Taking the derivative again gives:
\[f''(z) = \frac{1}{2\pi i}  \oint_{\partial G} \frac{2}{(\xi - z)^3} f(\xi) d\xi\]

Taking the derivative again gives:
\[f'''(z) = \frac{1}{2\pi i}  \oint_{\partial G} \frac{2 \cdot 3}{(\xi - z)^4} f(\xi) d\xi\]

Using induction shows that
\[f^{(n)}(z) = \frac{1}{2 \pi i} \oint_{\partial G} \frac{n!}{(\xi - z)^{n+1}} f(\xi) d\xi\]

This gives us the corollary:

\begin{corollary}
    If $f$ is holomorphic under the assumption of Cauchy's Integral Formula, then $f$ is infinitely differentiable.
\end{corollary}

\begin{proof}
We note that for our claims about the derivative to work, we want to show that
$$\lim_{\Delta z \to 0} \int f(\xi) \frac{P(z + \Delta z, \xi)}{\Delta z} d\xi = \int \lim_{\Delta z \to 0} f(\xi) \frac{P(z + \Delta z, \xi) - P(z, \xi)}{\Delta z} d\xi$$
, ie. we can exchange the limit and the integral.\\

Then we have that
$$\int \lim_{\Delta z \to 0} f(\xi) \frac{P(z + \Delta z, \xi) - P(z, \xi)}{\Delta z} d\xi = \int f(\xi) \frac{\partial P}{\partial z}(z, \xi) d\xi$$

Indeed, we can exchange the limit and integral using \textbf{Dominated Convergence Theorem}.
\end{proof}

\begin{remark}
    Note that we were able to use the Dominated Convergence Theorem because of the following two reasons:
\begin{itemize}
    \item Reason 1:
\begin{lemma}
$L = \lim_{x \to x_0} f(x)$ exist if and only if for any sequence $\{x_n\}$ converging to $x_0$, 
\[\lim_{n \to \infty} f(x_n) = L\]
\end{lemma}
\item Reason 2:
\begin{theorem}[Mean Value Estimates]
$|P(z + \Delta z, \xi) - P(z, \xi)| \leq |\frac{\partial P}{\partial z}(z + \theta \Delta z, \xi)| \cdot |\Delta z|$, where $0 < \theta < 1$.\\\\
We note that this inequality is actually uniformly bounded
\[|\frac{\partial P}{\partial z}(z + \theta \Delta z, \xi)| \cdot |\Delta z| \leq M\]
This is because the domain of $\frac{\partial P}{\partial z}$ is compact and $\frac{\partial P}{\partial z}$ is continuous.\\\\
We also note that $|\partial G| < \infty$, so we can apply the Dominated Convergence Theorem.
\end{theorem}
\end{itemize}
\end{remark}

\begin{theorem}[Holomorphic Implies Analytic]
Let $\Omega \supset G$, where $\Omega$ is open and $G$ is a bounded open set where $\partial G \in PC^1$, and furthermore that $cl(G) \subset \Omega$. Let $z_0 \in G \subset \Omega$, then if $f \in CR^1(\Omega)$, then $f$ is analytic in $\Omega$. In particular,
$$f(z) = \sum_{n = 0}^\infty a_n (z - z_0)^n$$ for all $z$ where $|z - z_0| < dist(z_0, \partial G)$ and
\[a_n = \frac{1}{2\pi i} \int_{\partial G} \frac{f(\xi)}{(\xi - z)^{n+1}} d\xi = \frac{f^{(n)}(z_0)}{n!}\]
\end{theorem}

\begin{remark}
    In most textbook the neighborhood of power-series is over $|z - z_0| = R$ where $R < dist(z_0, \partial G)$
\end{remark}

\begin{proof}[Proof of Holomorphic Implies Analytic]
WLOG, we will assume $z_0 = 0$, because we can always shift the coordinate linearly.\\\\
\textbf{How does the WLOG} work, well, rewrite
\[\frac{1}{\xi - z} = \frac{1}{(\xi - z_0) - (z - z_0)}\]
, then we can just use the same reasoning as before. Trying to justify this takes a little bit of work.\\\\
Next, Cauchy's Integral Formula says
\[f(z) = \frac{1}{2\pi i} \int_{\partial G} \frac{f(\xi)}{\xi - z} d\xi\]
Now we note that
\[\frac{1}{\xi - z} = \frac{1}{\xi} \frac{1}{1 - \frac{z}{\xi}}\]
We note that $|\frac{z}{\xi}| < 1$ since $z$ is in the interior but $\xi$ is on the boundary, so
\[\frac{1}{\xi} \frac{1}{1 - \frac{z}{\xi}} = \frac{1}{\xi} \sum_{n = 0}^\infty \frac{z^n}{\xi^{n}}\]
So we have that
\begin{align*}
     \frac{1}{2\pi i} \int_{\partial G} \frac{f(\xi)}{\xi - z} d\xi &= \frac{1}{2\pi i} \int_{\partial G} \sum_{n = 0}^\infty \frac{f(z) z^n}{\xi^{n+1}} d\xi\\
     &= \sum_{n = 0}^\infty z^n (\frac{1}{2\pi i} \int_{\partial G} \frac{f(\xi)}{\xi^{n+1}} d\xi) \tag*{ASSUMING We can switch integral and sum}\\
     &= \sum_{n = 0}^\infty z^n a_n
\end{align*}
It remains for us to justify why we can switch this, we can usually use Fubini's Theorem or Dominated Convergence Theorem, but here we will just use a low-tech solution and note that, assuming $z$ is fixed and $\xi \in \partial G$, then
\[\lim_{N \to \infty} \sum_{n = 0}^N \frac{z^n}{\xi^{n+1}} = \frac{1}{\xi - z}\ \text{converges uniformly} \]
, because $|\frac{z}{\xi}| \leq \frac{|z|}{dist(z_0, \partial G)} < 1$, so we have this uniform convergence. Thus, we can always switch the integral and the sum.
\end{proof}

\begin{corollary}
    Under the same setup, if $f$ is analytic on $z_0$, then it converges in $D_{z, d}$ where $d = dist(z_0, \partial G)$, so the radius of convergence is at least as much as $d$.
\end{corollary}

\begin{remark}
    In general, uniform convergence always means that you can switch the limit and the integration. Indeed, let $S_n$ be a sequence of functions that uniformly converges to some function $S$, then we wish to show that
    \[\lim_{n \to \infty} \int S_n = \int \lim_{n \to \infty} S\]
    Indeed, since the convergence is uniform, for all $\epsilon > 0$, there exists some $N \in \Nbb$ such that for all $n > N$,
    \[|S_n(\epsilon) - S(\epsilon)| < \epsilon\]
    Taking the integrals gives then that
    \[|\int_{\partial G} S_n - \int S| \leq \epsilon |\partial G|\]
    Since this holds for any $\epsilon > 0$, the limit would thus converge.
\end{remark}

We have proved that satisfying Cauchy-Riemann implies analytic, we will now show the converse.

\begin{theorem}[Morera's Theorem]
If $f \in C^0(\Omega)$ and $\int_{\partial R} f(z) dz = 0$ for all sufficiently small rectnagles $R$ where $cl(R) \subset \Omega$, then $f \in CR^1$.
\end{theorem}

\begin{proof}
We will prove this next lecture.
\end{proof}

\begin{theorem}
$(\sum_{n = 0}^\infty a_n (z - z_0)^n)' = \sum_{n = 0}^\infty n a_n (z - z_0)^{n-1}$, hence power series are holomorphic, and thus analytic functions are holomorphic.
\end{theorem}

\begin{proof}
You can usually do it by taking the partial sum and taking limit, but there's a trick to it! This is a direct corollary of Morera's Theorem, more in the next lecture.
\end{proof}