\section{Lecture 8}

\subsection{Homotopy Invariance of Integrals - Continued}

\begin{theorem}
If $\gamma_0, \gamma_1$ are homotopy equivalent in $\Omega$ and $f \in Hol(\Omega)$, then
\[\int_{\gamma_0} f(z) dz = \int_{\gamma_1} f(z) dz\]
\end{theorem}

\begin{proof}
Let $\Gamma$ be the homotopy equivalence function. For all $z_0 \in \Omega$, let $D_{z_0}$ be a disc centered at $z_0$ small enough such that $f$ can be expressed as a power-series on it, define
\[U_{z_0} = \Gamma^{-1}(D_{z_0})\]
Then $\{U_{z_0}\}$ will be an open cover of $R \coloneqq [a, b] \times [0, 1]$, which is a compact metric space. Apply Lebesgue's Number Lemma, we find $\delta > 0$ such that, we can split $R$ into smaller rectangles $R_k$ where each $diam(R_k) < \delta$.\\\\
The Lebesgue Number Lemma tells us that, for each $R_k$, $\Gamma(cl(R_k)) \subset D_{z}$ for some $z$. Since $f$ is primitive on $D_z$, we have that
\[\int_{\Gamma|_{\partial R_k}} f(z) dz = 0\]
Putting the small rectangles together (Green's Theorem style), the inner edges cancel out, so we have that
\[\int_{\Gamma|_{\partial R}} f(z) dz = 0\]
Now consider the pathes $\ell_1 = \{a\} \times [0, 1]$ and $\ell_2 = \{b\} \times [0, 1]$, we claim that
\[\int_{\Gamma|_{\ell_1 + \ell_2}} f(z) dz = 0\]
Indeed, if endpoints are fixed, both paths are constant (stays at same point). If all path are closed, then this is the same path in opposite directions.\\\\
Thus, we only have to integrate over the horizontal pathes:
\[\int_{\partial R} f(z) dz = 0 = \int_{\gamma_0} f(z) dz - \int_{\gamma_1} f(z) dz\]
\end{proof}

\begin{definition}
    A set $\Omega$ is \textbf{simply connected} if any closed loop in $\Omega$ is homotopy equivalent to the constant path. This is equivalent to saying, for any pathes $\gamma_0, \gamma_1$, $\gamma_0(a) = \gamma_1(a) = z_0$, $\gamma_0(b) = \gamma_1(b) = z_1$, are homotopy equivalent.
\end{definition}

\begin{remark}
If $\Omega$ is simply connected domain, then
\[\int_{z_0}^{z_1} f(z) dz\]
does not depend on the path specified.
\end{remark}

\begin{definition}
    Let $\Omega$ be simply connected and $f \in Hol(\Omega)$. Suppose $f(z) \neq 0$ for all $z \in \Omega$.\\\\
    Fix some $z_0 \in \Omega$, and consider
    \[a_0 \text{ such that } f(z_0) = e^{a_0}\]
    We can find $a_0$ as
    \[Re(a_0) = \log |f(z_0)|, Im(a_0) = arg\ f(z_0)\]
    Then we define
    \[\log f(z) \coloneqq a_0 + \int_{z_0}^z \frac{f'(\xi)}{f(\xi)} d\xi\]
\end{definition}

Show that $\log f(z)$ aligns with the global definition of the complex logarithm.

\begin{proof}
Exercise. The idea is to denote $\varphi(z) = a_0 + \int_{z_0}^z \frac{f'(\xi)}{f(\xi)} d\xi$ and show that
\[(f(z) e^{-y(z)})' = 0\]
and realize that their product must be $1$.
\end{proof}

\mattie{explain motivation later}
\begin{definition}
    What is $f(z)^{\alpha}$? We define
    \[f(z)^{\alpha} = e^{\alpha \log(f(z))}\]
    This is defined when $f \in Hol(\Omega)$, $f(z) \neq 0$ on all of $\Omega$, and $\Omega$ is simply connected.
\end{definition}

In a simply connected domain, the branch of the logarithm, it is enough to be determined by which $a_0$ you choose.

\subsection{Laurent Series}

Suppose $f$ is holomorphic on $\{z : a < |z - z_0| < A\}$, where we require $a \geq 0$ and $A \leq \infty$.

\begin{theorem}
If $f \in Hol\{z : a < |z - z_0| < A\}$, then $f$ can be represented as
\[f(z) = \sum_{n \in \Zbb} a_n (z - z_0)^n\]
, where we have that
\[a_n = \frac{1}{2\pi i} \int_{|\xi - z_0| = r} f(\xi) (\xi - z_0)^{-(n+1)} d\xi\]
, where $r$ is between $a < r < A$ (Note that for $\Zbb_+$, this is the same as the Taylor Power Series)
\end{theorem}

\begin{proof}
Without loss of generality, we can shift this to $z_0 = 0$.\\\\
Take $z$ such that $a < |z| < A$ and pick $r, R$ such that
\[a < r < |z| < R < A\]
Then consider the set
\[G \coloneqq \{z: r < |z| < R \}\]
Then Cauchy's Integral Formula gives
\begin{align*}
    f(z) &= \frac{1}{2\pi i} \int_{\partial G} \frac{f(\xi)}{\xi - z} d\xi\\
    &= \frac{1}{2\pi i} (\int_{|z| = R} \frac{f(\xi)}{\xi - z} d\xi - \int_{|z| = r} \frac{f(\xi)}{\xi - z} d\xi)\\
    &= \frac{1}{2\pi i} (I_1 - I_2)
\end{align*}
For $I_1$, we note that $|z| < |\xi| = R$, so
\begin{align*}
    I_1 &= \int_{|z| = R} \frac{f(\xi)}{\xi - z} d\xi\\
    &= \int_{|z| = R} \frac{f(\xi)}{\xi} \frac{1}{1 - \frac{z}{\xi}} d\xi\\
    &= \int_{|z| = R} f(\xi) \sum_{k = 0}^\infty \frac{z^k}{\xi^{k+1}}
\end{align*}
For $I_2$, we note that $|z| > |\xi| = r$, so
\begin{align*}
    I_2 &= \int_{|z| = r} \frac{f(\xi)}{\xi - z} d\xi\\
    &= \int_{|z| = r} \frac{-f(\xi)}{z} \frac{1}{1 - \frac{\xi}{z}} d\xi\\
    &= \int_{|z| = r} -f(\xi) \sum_{k = 0}^\infty \frac{\xi^k}{z^{k+1}}\\
    &= \int_{|z| = r} -f(\xi) \sum_{n = -\infty}^{-1} \frac{z^n}{\xi^{n+1}}\\ \tag*{Change of Variables}
\end{align*}
Since both series converge uniformly, we can integrate term by term and use Cauchy's Formula for Derivatives, rewriting both $I_1$ and $I_2$ out in series finishes the proof.
\end{proof}

Now we will consider a particular case of the Laurent Series!

\begin{definition}
    Let $f \in Hol(\Omega \setminus \{z_0\})$ where $z_0 \in \Omega$. In this case, we say $f$ has a singularity at $z_0$.\\\\
    Take $\delta$ small enough and $D_{z_0, \delta} \subset \Omega$, then the Laurent Series tells us
    \[f(z) = \sum_{n \in \Zbb} a_n (z - z_0)^n\]
    We say that
    \begin{itemize}
        \item $f$ has a \textbf{removable singularity} at $z_0$ if $a_n = 0$ for all $n < 0$. In this case,
        \[f(z) = \sum_{n \geq 0} a_n (z - z_0)^n, f(z_0) = a_0\]
        An example of such function would be $f(z) = \frac{sin(z)}{z}$ has removable singularity at $0$
        \item We say $f$ has a \textbf{pole at $z_0$} if $a_n \neq 0$ for finitely many $n < 0$. We say the \textbf{order of a pole} as
        \[\max \{n \geq 0: a_{-n} = 0\}\]
        \item We say that $f$ has an essential singularity if there exists infinitely many $n < 0$ such that $a_n \neq 0$.\\\\
        For example consider
        \[f(z) = e^{1/z} = \sum_{n = 0}^\infty \frac{z^{-n}}{n!}\]
        This has an essential singularity at $z = 0$.
    \end{itemize}
    These are the singularities we care about.
\end{definition}