\section{Lecture 3}

\subsection{Curves and Orientations}

\begin{definition}[Orientation of a Curve]
Let $\Gamma \subset \Cbb$ be a curve, and let $\gamma, \gamma_1: [a, b] \to \Cbb$ be the standard injective parameterization with $\gamma'(t) \neq 0, \gamma_1'(t) \neq 0$. Then we say $\gamma$ and $\gamma_1$ have the same orientation if $(\gamma \circ \gamma_1^{-1})' > 0$.
\end{definition}

Note that this is just the standard definition of orientation for $1$-dimensional manifolds.

On a closed loop, the \textbf{positive} direction is given by the \textbf{left-leg rule}, meaning tracing along the curve, the left leg of the curve points into the region enclosed.

This turns out to align exactly with the definition of orientation inherited by the boundary manifold.

\begin{definition}
We say $f \in CR^1(\Omega)$ if $f \in C^1(\Omega)$ and $\frac{\partial}{\partial \overline{z}} f = 0$ on $\Omega$.
\end{definition}

\begin{remark}
    It turns out that $CR^1 = C^1_z$, meaning that being in $CR^1$ is equivalent to being complex differentiable. The direction from $C^1_z \implies CR^1$ is shown in the previous lecture, what about the other direction?
\end{remark}

\begin{definition}[Multivariable Differentiability]
We say $f: \Omega \subset \Rbb^n \to \Rbb^m$ is differentiable at $x \in \Omega$ if there exists some $L \in M_{n \times m}(\Rbb)$ such that
\[f(x + h) = f(x) + L(h) + r_x(h)\]
, where $r_x(h)$ is sometimes denoted as $O(h)$ and $\lim_{h \to 0} \frac{r_x(h)}{||h||} = 0$
\end{definition}

\begin{proposition}
    If a function $f: \Omega \subset \Cbb \to \Cbb$ is $CR^1$, then $f$ is $C_z^1$
\end{proposition}

\begin{proof}
Since $f$ is $CR^1$, it is $C^1(\Omega)$, so $f$ is differentiable when viewed as a function in $\Rbb^2$. Write $f = u + iv$, then the differential of $f$ is exactly the Jacobian matrix:
\[df = \begin{pmatrix} \frac{\partial u}{\partial x} & \frac{\partial u}{\partial y}\\
\frac{\partial v}{\partial x} & \frac{\partial v}{\partial y} \end{pmatrix}\]
Then $f$ satisfiying the $CR$-equations tells us that
\[\frac{\partial u}{\partial x} = \frac{\partial v}{\partial y},\ \frac{\partial v}{\partial x} = -\frac{\partial u}{\partial y}\]
This means that $df$ is just a scaled rotational matrix, so it's without loss a multiplication by complex numbers. Let $a$ be the complex number representation of $df$.\\\\
Thus, we have that
\[f(z + h) = f(z) + a \cdot h + O(h)\]
Then we have that
\[\lim_{h \to 0} \frac{f(z + h) - f(z)}{h} = a\]
, so the complex derivative exists, so $f$ is holomorphic on $\Omega$.
\end{proof}

When we say something is differentiable on an non-open set $K$, then this means there exists some open $\Omega \supset K$ such that it is differentiable on it, meaning there's some bigger open set this function is differentiable on.

\subsection{Cauchy's Integral Formula}

\begin{theorem}[Cauchy Formula]
Let $G$ be a bounded open set with boundary $\partial G \in PC^1$. Let $f \in C^1_z(cl(G)) = CR^1(cl(G))$. Then for all $z \in int(G)$,
\[f(z) = \frac{1}{2 \pi i} \oint_{\partial G} \frac{f(\xi)}{\xi - z} d\xi\]
\end{theorem}

\begin{proof}
Let $\Omega \supset cl(G)$ be the function $f \in CR^1(\Omega)$. Now consider
\[g(z) = \frac{f(z)}{z - z_0}\]
Then we note that $\frac{1}{z - z_0} \in CR^1(\Cbb \setminus \{z_0\})$. Therefore, $g(z) \in CR^1(\Omega \setminus \{z_0\})$.\\\\
Choose $\epsilon > 0$ small enough such that $D_{z_0, \epsilon}$, the disk of radius $\epsilon$ centered at $z_0$, does not intersect with $\partial G$. Now consier $G_\epsilon \coloneqq G \setminus D_{z_0, \epsilon}$, then
\[\int_{\partial G_\epsilon} \frac{f(z)}{z - z_0} dz = 0 \forall \textbf{ small enough } \epsilon > 0\]
Now, we see that
\[\int_{\partial G_\epsilon} \frac{f(z)}{z - z_0} = \int_{\partial G} \frac{f(z)}{z - z_0} - \int_{\partial D_{z_0, \epsilon}} \frac{f(z)}{z - z_0}\]
Taking the limit of $\epsilon$ on both sides to $0$, so
\[0 = \int_{\partial G} \frac{f(z)}{z - z_0} - \lim_{\epsilon \to 0} \int_{\partial D_{z_0, \epsilon}}\frac{f(z)}{z - z_0} dz \]
\[\int_{\partial G} \frac{f(z)}{z - z_0} = \lim_{\epsilon \to 0} \int_{\partial D_{z_0, \epsilon}}\frac{f(z)}{z - z_0} dz\]
Since $f(z)$ is differentiable, we can write
\[f(z) = f(z_0) + f'(z_0) (z - z_0) + O(z - z_0)\]
So we have that
\begin{align*}
  \int_{\partial D_{z_0, \epsilon}}\frac{f(z)}{z - z_0} dz
  &= \int_{ |z - z_0| = \epsilon} \frac{f(z_0)}{z - z_0} dz + \int_{ |z - z_0| = \epsilon} f'(z_0) dz + \int_{ |z - z_0| = \epsilon} \frac{O(z - z_0)}{z - z_0} dz
\end{align*}
Parameterize $|z - z_0| = \epsilon$ by $z = z_0 + \epsilon e^{it}, t \in [0, 2\pi]$ and note that
\[\int_{|z - z_0| = \epsilon} \frac{1}{z - z_0} dz = 2 \pi i\]
\[\int_{ |z - z_0| = \epsilon} f'(z_0) dz = 0\]
\[|\int_{ |z - z_0| = \epsilon} \frac{O(z - z_0)}{z - z_0} dz| \leq 2 \pi \epsilon \max_{z \in |z - z_0| = \epsilon} |\frac{O(z - z_0)}{z - z_0}|\]
, which goes to zero as $\epsilon \to 0$. Then we have that
\begin{align*}
  \int_{\partial D_{z_0, \epsilon}}\frac{f(z)}{z - z_0} dz
  &= 2\pi i f(z_0) + \int_{ |z - z_0| = \epsilon} \frac{O(z - z_0)}{z - z_0} dz
  \lim_{\epsilon \to 0} \int_{\partial D_{z_0, \epsilon}}\frac{f(z)}{z - z_0} dz &= 2\pi i f(z_0) + 0
\end{align*}
Thus, we have that
\[\int_{\partial G} \frac{f(z)}{z - z_0} = 2\pi i f(z_0)\]
\end{proof}

Now we have that
\[f(z) = \frac{1}{2\pi i} \int_{\partial G} \frac{f(\xi)}{\xi - z} d\xi\]

We will see that shows that we can take the derivative infinitely many times and that $f(z)$ is analytic.