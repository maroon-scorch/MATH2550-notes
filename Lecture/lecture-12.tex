\section{Lecture 12 - 10/03/2022}
\subsection{Remark: Normal Convergence}

Note that the following facts holds on just in $\Cbb$, but in $\Rbb^n$ in general. Recall the two definitions of normal convergence:

\begin{definition}[Equivalent Definitions of Normal Convergence]
Let $\{f_n\}$ be a sequence of functions, we say $f_n \to f$ converes normally in $G$ if:
\begin{itemize}
    \item (a) for all compact $K \subset G$, $f_n$ converges to $f$ on $K$ uniformly.
    \item or, (b) if for all disc $D$ such that $cl(D) \subset \Omega$, $f_n$ converges to $f$ on $D$ uniformly.
\end{itemize}
The two definitions are in fact equivalent.
\end{definition}

\begin{proof}
$(a) \implies (b)$ is obvious. Now, for $(b) \implies (a)$, suppose $K \subset \Omega$ is compact, then for all $z \in K$, we can choose $D_z$ be an open disk centered at $z$ small enough that $cl(D_z) \subset \Omega$.\\\\
Then the collection $\{D_z\}_{z \in K}$ form a clear open cover of $K$, so we can find a finite subcover, $D_{z_1}, ..., D_{z_n}$. $f$ converges uniformly on each $cl(D_{z_i})$, so $f$ converges uniformly on $\bigcup_{k = 1}^n D_{z_i}$, which covers $K$.
\end{proof}

How do we in general check for normal convergence? It seems like we will have to check uncountably many disks!

\begin{proposition}[Strong Covering Property]
Consider the sequence $\{K_n\}_{n = 1}^\infty \subset \Omega$ such that for all compact $K \subset \Omega$, there exists some $N$ such that $K \subset \bigcup_{n = 1}^N K_n$.\\\\
%Note that this coniditon implies that the union of the $\{K_n\}$ is
Then, $f_n$ converges to $f$ uniformly on all compact sets $K \subset \Omega$ if and only if $f_n$ converges to $f$ uniformly on each $K_n$.
\end{proposition}

\begin{proof}
The forward direction is obvious. Conversely, for any compact set $K$, find $N \in \Nbb$ such that $K \subset \bigcup_{n = 1}^N K_n$. Then the uniform convergence on each $K_n$ implies uniform convergence on all finite unions on all of $\bigcup_{n = 1}^N K_n$, hence the uniform convergence is held on $K$.
\end{proof}

\begin{remark}
In practice, for open subset $\Omega \subset \Rbb^d$, we would define each $K_n$ as
\[K_n \coloneqq \{x \in \Omega\ |\ dist(x, \Omega^c) > \frac{1}{n}, |x| \leq n\}\]
\mattie{Ask about this}
\end{remark}

\begin{definition}
We say the collection $\{K_n\}_{n = 1}^\infty$ is a \textbf{compact exhaustion} of $\Omega$ if
\[K_n \subset int(K_{n+1}) \text{ and } \bigcup_{n = 1}^\infty K_n = \Omega\]
, note that the latter implies for any compact $K \subset \Omega$, there exist $N \in \Nbb$ such that $K \subset \bigcup_{n = 1}^N K_n$
\end{definition}

\subsection{Seminorms of Holomorphic Function}

\begin{definition}
Let $Hol(\Omega)$ be the set of all holomorphic functions on $\Omega$, given $f \in \Omega$, we define the norm
\[||f||_{C(K_n)} = |\sup_{z \in K_n} f(z)|\]
When $K_n$ are points, this is called a semi-norm. $C(K_n)$ refers to the space of continuous real-valued functions on $K_n$.
\end{definition}

\begin{proposition}
If $K_n$ has the strong covering property, then $f_n$ converges to $f$ on compact sets uniformly if and only if $||f_n - f||_{C(K_n)} \to 0$ as $j \to \infty$.
\end{proposition}

\begin{remark}
An important object of study in functional analysis are what's called \textbf{Frechet Spaces}, whose topology is given by countably many semi-norms.
\end{remark}

\begin{definition}
Let $f, g \in Hol(\Omega)$, we can define a metric on $Hol(\Omega)$ as
\[p(f, g) = \sum_{n = 1}^\infty \frac{1}{2^n} \frac{||f - g||_{C(K_n)}}{1 + ||f - g||_{C(K_n)}}\]
One can verify that this is indeed a metric (more details in Conway's Book)
\end{definition}

\begin{proposition}
$f_n$ converges to $f$ uniformly on compact sets if and only if $p(f_n, f) = 0$ as $n \to \infty$.
\end{proposition}

In practice, no one would actually compute the norm or fixes a compact exhaustion. However, there are great theoretical values in this metric.

\begin{theorem}
$(Hol(\Omega), p)$ is a complete metric space.
\end{theorem}

\begin{proof}
We first note that for a compact set $K$, $C(K)$ is a complete normed space (hence a Banach Space), so we observes that
\begin{itemize}
    \item 1. $\lim_{k \to \infty} p(f_k, f) = 0$ if and only if $\lim_{k \to \infty} ||f - f_k||_{C(K_n)} \to \infty$ for all $n$.
    \item 2. $\{f_k\}$ is Cauchy with respect to $p$ on $Hol(\Omega)$ if and only if $f_k$ is Cauchy with respect to $||\cdot||_{C(K_n)}$ on $C(K_n)$ for every $n$ 
\end{itemize}
Now, every convergent sequence is Cauchy. Conversely, suppose $\{f_k\}$ is a Cauchy sequence, then since each $C(K_n)$ is a complete normed space, Hence on each $K_n$, $\{f_k\}$ converges uniformly to some limit $F_n$.\\\\
We claim that we can actually find a global function $f \in C(\Omega)$ such that $f|_{K_n} = F_n$. Indeed, we want to essentially glue each $F_n$ together, which follows from the glueing lemma that for any two continuous function $f$ on $K_1$ and $g$ on $K_2$ (within $\Cbb$ so Hausdorff) with non-empty intersection $K_1 \cap K_2$, we can extend $f$ and $g$ together into a larger continuous function.\\\\
Now, since $f$ is the limit on each $K_n$, we have that $||f_k - f||_{C(K_n)} \to 0$ as $k \to \infty$, for all $n$. Hence we have from Observation $1$ that $\lim_{k \to \infty} p(f_k, f) = 0$, so $f$ is the limit of the Cauchy seuqnece in $Hol(\Omega)$.\\\\
It remains for us to show that $f \in Hol(\Omega)$. Indeed, we note that for any $K \subset \bigcup_1^N K_j$, $f_n$ converges to $f$ uniformly, hence by Morera's Theorem, we can get that $f \in Hol(\Omega)$.
\end{proof}

\subsection{Open Mapping Theorem and Inverse Function Theorem}

Suppose $f \in Hol(z)$ (non-constant), $z_0 \in \Omega$, $f(z_0) = w_0$. Let $G$ be a bounded domain with $z_0 \in G$ and $\partial G \in PC^1$.\\\\
Furthermore, for all $z \in \partial G$, suppose we have $|f(z) - w_0| \geq \delta > 0$.\\

Does such $G$ always exist? Yes!

This is because the zeroes of the function $f(z) - w_0$ are isolated (as $f(z)$ is non-constant). Therefore, $f(z) - w_0$ has no zeroes in $cl(D_{z_0, r}) \setminus \{z_0\}$ for a sufficiently small $r$.\\

Let $w \in \Cbb$ such that $|w - w_0| < \delta$, and let $m$ be the multiplicity of the zero $z_0$ of $f(z) - w_0$

\begin{lemma}
Fix $w$ as above, the fiber $f^{-1}(w)$ has $m$ points counting multiplicity in $G$, meaning that the function $g(z) = f(z) - w$ has zeroes $z_1, ..., z_m$ (counting multiplicity)
\end{lemma}

\begin{proof}
We will apply Rouche's Theorem on this. Indeed, let $g_0(z) = f(z) - w_0$, then we note that
\[g(z) = f(z) - w = [f(z) - w_0] + [w - w_0]\]
By our setup above, we know that $|f - w_0| \geq \delta$ and $|w - w_0| < \delta$. So Rouche's Theorem tells us that both $f(z) - w_0$ and $g(z) = [f(z) - w_0] + [w - w_0]$ have the same number of zeroes, counting multiplicity.\\\\
$m$ was defined to be exactly the multiplicity of $z_0$ at $f(z) - w_0$, as we chosen $G$ small enough that $f(z) - w_0$ has no other zeroes within. 
\end{proof}

\begin{corollary}[Open Mapping Theorem]
If $f \in Hol(\Omega)$ be a non-constant function, then $f$ is an open map, meaning that for all open sets $U \subset \Omega$, $f(U)$ is open.
\end{corollary}

\begin{proof}
The argument here will run similarly to how we had above.\\\\
Let $w_0 \in f(U)$, we want to show that there exist some open set $V \subset f(U)$ that contains $w_0$.\\\\
Since $w_0 \in f(U)$, we know there exist some point $z_0 \in U$ such that $f(z_0) = w_0$. Since $U$ is open, we can find some radius $\epsilon > 0$ small enough that $cl(D_{z_0, \epsilon}) \subset U$.\\\\
Now consider the function $g(z) = f(z) - w_0$. Since $g$ is holomorphic and non-constant, the uniqueness theorem for analytic function tells us that the zeros of $g(z)$ are isolated. Hence, we can choose $\epsilon > 0$ small enough that without loss $g(z)$ only has the root $z_0$ in $cl(D_{z_0, \epsilon})$.\\\\
Since $\partial D_{z_0, \epsilon}$ is compact, and $|g(z)|$ is continuous and positive, the extreme value theorem shows that there exist some minimum value $\delta > 0$ such that $\delta$ is the minimum of $|g(z)|$ for $z$ on the boundary.\\\\
Now consider the disk $D_{w_0, \delta}$. For any $w \in D_{w_0, \delta}$, by Rouche's Theorem, write $g(z) = f(z) - w = [f(z) - w_0] + [w - w_0]$, then we note that $|g(z)| \geq \delta > |w - w_0|$, then this means that $f(z) - w_0$ and $g(z)$ have the same number of zeroes in the disk $D_{z_0, \epsilon}$.\\\\
In particular, this means that there exist some $z \in D_{z_0, \epsilon}$ such that $f(z) = w$. So in other words, $w \in f(U)$.\\\\
Hence, $D_{w_0, \delta} \subset f(U)$. In other words, $w_0 \in D_{w_0, \delta} \subset f(D_{z_0, r}) \subset U$, which implies that $f(U)$ is open.
\end{proof}

\begin{theorem}[Inverse Mapping Theorem]
Let $f \in Hol(\Omega)$, $z_0 \in \Omega$, and $f(z_0) = w_0$ but $f'(z_0) \neq 0$ (so the zero is simple). Let $G \subset \Omega$ be an open bounded set that contains $z_0$ with piecewise $C^1$ boundary.\\\\
Now suppose that, for all $z \in \partial G$, $|f(z) - w_0| \geq \delta$, then for all $w$ such that $|w - w_0|$, there exists some unique function $z$ such that $z = f^{-1}(w)$ and
\[f^{-1}(w) = \frac{1}{2\pi i} \int_{\partial G} \frac{\xi f'(\xi)}{f(\xi) - w} d\xi\]
\end{theorem}